%*******************************************************************************
%*********************************** First Chapter *****************************
%*******************************************************************************

\chapter{Introduction} 


Vehicle makes our life easier by bringing us travel convenience, 
but it also produces problems such as travel cost, carbon pollution, 
car accident and road congestion.
In 2015, about 140.43 billion gallons of gasoline were consumed in the United States. 
Also, according to the US Census Bureau, there are about ten million car accidents every year.
It has been shown that most car accidents are mainly 
caused by dangerous driving behaviors \cite{progressive}, such as distracted and/or aggressive driving.  
These problems are caused by the lack of understanding, monitoring 
and careful control on vehicle dynamics. 
Better understanding and control on vehicle dynamics
can not only reduce car accidents \cite{progressive}, 
but also improve fuel efficiency \cite{morganstanley2013}. 


There has been active efforts to achieve better understanding on
human driving behaviors and better control on vehicles. 
One such example is to use driverless car system to 
replace human drivers \cite{googledriverlesscar, kumar2012carspeak,
urmson2008autonomous,litman2013autonomous}. 
However, there are many unresolved regulatory, legal and insurance issues, 
so it is not feasible to adopt driverless car in short and medium term.   
We believe that using existing technologies to assists human drivers
in user-operated vehicles are more critical for safe and efficient driving activities \cite{you2013carsafe, wang2013sensing, chen2015invisible, uber}. 
There are smartphone sensing applications that 
monitor driving behaviors and sensing vehicle's movement \cite{you2013carsafe, wang2013sensing},
Also, there are in-vehicle systems that assist human drivers with some driving functionalities, 
e.g., cruise control \cite{bengtsson2001adaptive, cruise_control} and emergence braking system \cite{emergency_brake} etc. 
However, there are many questions that have not been answered: 
what is the accuracy of smartphone sensors to capture driving behaviors?
how to distinguish driver and passenger with zero infrastructure?
how to assist human drivers to achieve fuel efficient and safe driving?


With many open questions in mind, we ask a high-level question in this proposal: 
\emph{how can we build sensing and control system blocks for modern vehicles to
monitor, assist or even replace human drivers in
improving driving performance, i.e., enhancing fuel efficiency and improving road safety.}
To answer this question, we explore the sensing and control capabilities
of commodity hardwares and how to utilize on modern vehicles. 
For example, we can use smartphone to sense and capture 
various driving behaviors, 
based on which we can evaluate driving performance. 
We can also access vehicle parameters from OBD 
port \cite{obd} to evaluate our driving behavior detection algorithms. 
We implemented a smartphone application called DriveSense to collect
data and verify our algorithms. 
We also designed and implemented a control system called EcoDrive that 
leverages drive-by-wire technology to control fuel injection
rate to adjust speed for higher fuel efficiency. 



\textbf{Motion and Driving Activity Sensing}. 
Capturing human driving activities (e.g., brakes) are important
in lots of road safety applications \cite{uber}, 
as it has long been believed that there is close relation
between driving behavior and road safety \cite{progressive}.
Smartphone becomes an ideal platform due to its rich built-in
sensors, i.e., accelerometer, gyrosocope and vision sensors etc. 
We develop smartphone application, called DriveSense 
to monitor human drivers by using smartphone sensors, 
i.e., detecting driving events and monitoring driver distractions etc. 
We design a slope-aware method to estimate vehicle motion by using
smartphone IMU sensors. 
It addresses the challenges for coordinate alignment between the smartphone 
and the car and linear acceleration estimation under various sloping road conditions.
We further design a smartphone application, called DriveSense, 
to monitor driving behaviors and driver distractions by using various sensors on smartphone, 
i.e., IMU sensors, GPS and vision sensors. 
To evaluate our solutions in larger scale, we implement an Android
application called DriveSense, that will be released in Google
App store to benefit more users. 
We conduct controlled experiments to understand the problems, 
evaluate our solutions and debug the systems. 
We also conduct uncontrolled experiments to collect data
for analysis and evaluate our solutions in the wild. 




\textbf{Vehicle Access and Control}. 
Direct access to vehicular parameters provides us the ground truth data
that can be used to verify our models and algorithms. 
We can access the speed data from the vehicle OBD 
port to verify speed estimation algorithm. 
We can also access the fuel usage data to estimate 
fuel efficiency. 
With the control the vehicle, we can assist human drivers to 
achieve higher fuel efficiency (e.g., smart cruise control system) 
and safer driving (e.g., emergency brake system). 
EcoDrive is one such example system we built. 
EcoDrive is a fuel consumption monitoring and control system that can  
improve fuel efficiency and reduce carbon emissions. 
EcoDrive models acceleration and instant fuel consumption of individual vehicles
and emulates gas pedal position sensor 
(replace gas pedal pressed by human drivers) to control vehicle accelerations. 
To evaluate our driver assistance systems, we use various methodologies such as 
data analysis, system implementation and road test. 
To model vehicle's fuel consumption and capture driving behaviors,
we collected 5,000 miles urban driving data and 5,000 miles highway
driving data from 12 different drivers. 
We also build system prototype that is installed
on regular vehicle to evaluate its performance. 




We will describe our driver assistance
systems and proposed further work in high level 
in the rest of this chapter. 
The detailed design, analysis, implementation and evaluation of 
each system are presented in details in following chapters. 


\section{Estimating Vehicle Accelerations by Using IMU Sensors}

To better understanding driving behaviors and potentially
improve road safety, 
one can use smartphone IMU sensors to estimate vehicle motion, driving
performance evaluation and real-time warning on aggressive
events, e.g., hard brakes or rough accelerations. 
However, it is not an easy task, 
i.e., we found that even gentle road slopes may cause 
coordinate misalignment and acceleration over- or under-estimation
if accelerometer is used. 
It is an important problem
to address, given that no road is going to be perfectly level
— it may slope with a small grade even if it appears level
to the eye. Our modules solves the problem to continuously
estimating the slope of the road (and of the vehicle) and using
this information to more accurately infer true vehicle motion
parameters, (e.g., acceleration). In particular, we solve three
related problems. First, even a gentle grade can cause over-
or under-estimation of brakes and accelerations as detected
by inertial sensors. Second, road slopes introduce noise on
coordinate alignment between the smartphone and the vehicle,
and with existing techniques it takes a long time to converge to an
accurate value. Third, the accumulated gyroscope error leads to
inaccurate slope estimations. The error source includes hardware
limitations and affected component from other dimensions due
to (even small) misalignment. We implement our solution as a
software module for smartphones and evaluate its performance
using traces from drives across 10,000 miles of highway and
urban roads in the US. Compared to the state of the art, our
estimation method is more accurate and efficient — our slope-
aware coordinate alignment algorithm achieves 3 times faster
on median convergence time and linear acceleration estimation
method improves linear acceleration estimation accuracy more
than 2 times on average.

\section{Improving Fuel Efficiency and Reducing Carbon Emissions}

We designed a control system called EcoDrive to improve fuel efficiency
by modeling fuel consumption and emulating gas pedal position sensor. 
EcoDrive senses vehicle dynamics through
the standard vehicle On-board diagnostics (OBD) port and 
models various vehicle forces, i.e., 
propulsion, drivetrain loss, wind resistance and grade resistance, 
as functions of instant fuel consumption. 
By sensing vehicular speed and controlling air/fuel injection rate in real time, 
EcoDrive can adjust speed carefully to improve fuel efficiency by
emulating the gas pedal position sensor. 
We collected more than 10,000 miles of driving traces from 12 different vehicles
to build models of vehicle dynamics. 
Based on the models, a prototype of EcoDrive is implemented in an off-the-shelf 
embedded platform. 
The prototype is installed on a 2011 Chevy Impala and 
evaluated through test drives of more than 100 miles across both
urban and highway environments. 
In comparison with human drivers, 
EcoDrive achieves an average of 20\% higher fuel efficiency in urban road segments 
and 30\% higher fuel efficiency on highways. 






