%*******************************************************************************
%*********************************** First Chapter *****************************
%*******************************************************************************

\chapter{Introduction} 


Vehicle makes our life easier by bringing us travel convenience, 
but it also produces problems such as travel cost, carbon pollution, 
car accident and road congestion.
In 2015, about 140.43 billion gallons of gasoline were consumed in the United States. 
Also, according to the US Census Bureau, there are about ten million car accidents every year.
It has been shown that most car accidents are mainly 
caused by dangerous driving behaviors and human mistakes \cite{progressive}. 
These problems are caused by the lack of understanding, monitoring 
and careful control on vehicle dynamics. 
Better understanding and control on vehicle dynamics
can not only reduce car accidents \cite{progressive}, 
but also improve fuel efficiency \cite{morganstanley2013}. 


There has been active efforts to achieve better understanding on
human driving behaviors and better control on vehicles. 
One such example is to use driverless car system to 
replace human drivers \cite{googledriverlesscar, kumar2012carspeak,
urmson2008autonomous,litman2013autonomous}. 
However, it takes time for self-driving systems to be robust enough to 
replace traditional vehicles. 
Also, it is unlikely that self-driving
systems are going to ever achieve perfect accuracy under all
conditions.
Meanwhile, some researchers are using existing technologies to assists human drivers
in user-operated vehicles are more critical for safe 
and efficient driving activities \cite{you2013carsafe, wang2013sensing, chen2015invisible, uber}. 
Also, there are in-vehicle systems that assist human drivers with some driving functionalities, 
e.g., cruise control \cite{bengtsson2001adaptive, cruise_control} and emergence braking system \cite{emergency_brake} etc. 
However, there are many questions that have not been answered: 
what is the accuracy of smartphone sensors to capture driving behaviors?
how to assist human drivers to achieve fuel efficient and safe driving?
how to make self-driving systems more reliable? 


With many open questions in mind, we ask a high-level question in this proposal: 

\emph{how can we build sensing and control system blocks for modern vehicles to
monitor, assist or even replace human drivers in
improving driving performance and experience.}


To answer this question, we explore the sensing and control capabilities
of commodity hardwares and how to utilize them on modern vehicles. 
For example, we can use smartphone to sense and capture 
various driving behaviors, 
based on which we can evaluate driving performance. 
We can also access vehicle parameters from OBD 
port \cite{obd} to evaluate our driving behavior detection algorithms. 
We implemented a smartphone application called DriveSense to collect
data and verify our algorithms. 
We also present a control system called EcoDrive that 
leverages drive-by-wire technology to control fuel injection
rate to find a tradeoff between travel time and fuel efficiency. 
To handle occational self-driving system failures, 
we present a live streaming and remote control framework
called RTDrive to augment self-driving systems. 


\section{DriveSense}



Smartphone-based sensing has opened up a whole gamut of applications and services
across many domains. In the world of the transportation system, it is being increasingly used
for crowd-sourcing various forms of information, ranging from road and traffic conditions to data on traffic light patterns,
and even interesting annotations by participants.
One such application is the ability to independently monitor driving behavior --- how well is one
driving the vehicle, e.g., aggressive driving actions, such as rough acceleration, hard brakes, sharp turns, and more.
The popularity of smartphones enable innovative ways of monitoring such behaviors and actions.
For example, Cambridge Mobile Telematics \cite{cmt} 
develops a smartphone
app to capture driving behaviors and monitor driver
distractions.
Similarly, the well-known ride-sharing company Uber announced
it will start tracking Uber drivers' driving behaviors
with their smartphones and give them feedbacks that are more detailed than 
the five-star rating customers leave for each driver \cite{uber}.
Another set of applications that can leverage sensing of motion parameters is to understand road conditions, to
say, detect potholes \cite{eriksson2008pothole} or to detect icy stretches during a bad snow event.

A common technique to detect various vehicle parameters, especially acceleration, braking, and turns ---
events which happen in short timescales --- is to use built-in inertial measurement unit (IMU) 
sensors available in the mobile devices \cite{wang2013sensing,hansenspeed,chen2015invisible}. 
The general approach taken by such systems is to measure the three-dimensional acceleration
using the accelerometer and to measure the three-dimensional relative rotation speed using
a gyroscope, including various de-noising techniques to make precise estimates.
An important step in such design is one of coordinate alignment where the system needs to
calculate the rotation matrix that translates 
the coordinates of the smartphone to that of the vehicle. 



However, we find that exsiting approaches may lead to significant estimation
errors (especially acceleration) if 
the smartphone is not tightly mounted, there are relative orientation changes,
and/or the vehicle is moving over road slopes. 
If the mobile device is not tightly fixed
in a mount, the device may move occasionally within the mount and its orientation 
relative to the vehicle may frequently change causing measurement
errors. Worse, if the device is occasionally held in a person's hand, the update process needs
to be applied continuously significantly exacerbating the challenge. Finally,
even if the device is perfectly fixed to a mount, there are potential errors since the device itself
might not be able to determine whether the vehicle itself is on flat earth or is 
tilted due to the slope of a road.





\nop{
As a consequence, it is worth considering whether an alternative approach to detecting these
motion parameters is possible through the use of GPS.
Clearly, GPS is widely used in many outdoor applications to identify location and even speed,
and acceleration can be determined by observing changes in speed.
Use of GPS has one big advantage --- we do not need to do orientation corrections between
device and vehicle and makes the computation both simpler and more robust.
However, low-cost GPS receivers that are available in common mobile devices have other problems.
They do not always have
sufficient resolution to accurately estimate some short timescale events, e.g., rotation
speeds, acceleration and brakes.
Further, such GPS devices do not always have the greatest accuracy and resolution in location
estimation, 
which may also lead to additional errors in estimating acceleration
and brakes.
}


\nop{
In this paper, we therefore, first, conduct a study of whether some of the vehicular motion
parameters of interest, especially those related to acceleration, brakes, and turns, 
can be estimated accurately when there are slopes.
To understand the performance of GPS and inertial sensors for these purposes,
we use measured data from vehicles collected through more than 10,000 miles of
motion, 
 which provide GPS data and inertial sensor measurements from mobile devices and 
 some independently logged speed data from the On-board Diagnostics (OBD)
interfaces of the same vehicle.
}
 
%Based on our evaluation of GPS accuracy over more than 10,000 miles of 
%driving data


To understand and improve the performance of inertial sensors, 
we develop several 
techniques to detect orientation changes, model mounting stability, 
and improve coordinate alignment accuracy. 
We model the orientation of the smartphone as a cluster
of accelerometer data and 
use moving variance to detect possible orientation changes. 
We use \emph{intra-cluster variance} (ICV) to model mounting stability 
of the smartphone and use x-percentile of the variance 
as an indicator of motion estimation accuracy. 
Our intuition is that loose placement introduces more noises 
into sensor data and leads to less accurate acceleration estimation. 
We also find that existing coordinate alignment techniques produce
less accuracy when there are (even small) slopes due to the gravity
components sensed by accelerometer. 
We use an example trip to show that slope may cause coordinate misalignment 
and acceleration over/under estimation. 
Therefore, we use IMU sensors to tracked the slope gradients and subtract
the gravity components sensed by accelerometer. 


We conclude that the performance of inertial sensors are constrained due to
road conditions or human interactions and only well-tuned algorithm can produce higher accuracy 
than GPS in low speed scenarios, 
while slope has limited impact on steering motion estimation performance by gyroscope. 
In addition, we found that, the $90$-percentile accuracy of GPS is under $0.5m/s^2$, 
which indicate a better accuracy than accelerometer on average. 
In particular, we find that GPS is more accurate in high speed scenarios and 
only well-tuned inertial sensors can achieve comparable accuracy. 


The second goal of this paper is to use the above observations to develop a combined system
to conduct driving analytics, called DriveSense, a mobile application that achieve
higher estimation accuracy due to slope awareness, and leverages GPS to complement
inertial sensors, when the latter lacks necessary accuracy and not available. 
More specifically,
DriveSense is able to select the current best estimation 
based on confidence value from GPS and inertial sensors. 
By considering both inputs from GPS and inertial sensors, 
DriveSense has the $75th$ percentile error of $0.2m/s^2$, 
which is $5\times$ better than well-tuned inertial sensors in traditional approach. 





\section{EcoDrive}



The fuel conumption and carbon pollution caused by 
driving activities are drawing more and more attentions. 
In 2013, the White House issued a climate action plan to reduce fuel consumption
and carbon pollution \cite{whitehouse2013}. 
In the same year, Morgan Stanley reported that 
there will be \$158 billion annual savings in the US 
if all cars adopted smooth driving styles \cite{morganstanley2013}. 
We introduce a driver assistance system, called EcoDrive, 
that can improve the fuel efficiency of a vehicle's drive by sensing, computing, 
and actuating the acceleration behavior of the vehicle in an autonomous manner, 
by modeling properties of the vehicle, road conditions, and driving actions. 
With the global push for improving fuel efficiency of vehicles to 
reduce consumptions and carbon emissions, 
we believe solution such as ours can be one of many important mechanisms to meet such a goal.


\begin{figure}[t]
\begin{center}
\vspace{-0.5cm}
\includegraphics[width=4.0in,angle=0]{Figs/EcoDrive/architecture.pdf}
\vspace{-0.5cm}
\caption{EcoDrive Architecture.}
\vspace{-0.7cm}
\label{ecodrive}
\end{center}
\end{figure}


EcoDrive estimates instant fuel consumptions of different driving behaviors
based on sensed vehicle parameters from the On-board diagnostics (OBD) port \cite{obd, pid}. 
It can adjust vehicular speed in real time according to 
individual vehicle properties and road conditions 
to achieve higher fuel efficiency measured by Kilometer Per Liter (KPL)
\footnote{KPL refers to the distance travelled per unit volume of fuel consumed. 
It interchangeable with Mile Per Gallon (MPG), i.e., 1 KPL = 2.35214583 MPG.}.
EcoDrive is an independent system that can be installed on or removed from 
regular vehicles easily. 
This system controls the vehicle's acceleration and speed 
to provide a fuel efficient drive on its path. 
In our work and current implementation, 
we design this system assuming there is no other factors that 
would contribute to a choice of acceleration and speed, 
e.g., other vehicles, pedestrians, etc. or other obstacles in vicinity. 
Clearly in a practical system, this knowledge would be critical in modifying the acceleration behavior. 
Currently, we adopt the approach followed by other equivalent systems, 
such as cruise control \cite{cruise_control}, which allows the driver to instantly 
disable cruise control by actively pressing the brake pedal. 
In an analogous way, in our current implementation, 
we provide the driver a switch which can be pressed to instantly disable EcoDrive, 
if its acceleration behavior is perceived to be unsafe for nearby vehicles or obstacles.






\textbf{EcoDrive Components}. 
EcoDrive delivers its design via three components including an OBD sensing component, 
a vehicle dynamics modeling component and an acceleration controlling component. 
The architecture is illustrated in Fig. \ref{ecodrive}. 
The sensing component reads real-time OBD parameters through
the OBD port. 
The modeling component models various vehicle forces 
as functions of instant fuel consumption and produces
a fuel consumption profile, called Air/Fuel Rate (AFR) profile
\footnote{AFR refers to the volume of air/fuel cost per unit time.}.
The controlling component utilizes the AFR profile to calculate fuel efficient
driving strategies according to speed limit and road conditions. 
EcoDrive emulates the gas pedal by sending voltage values to 
the connector through an Arduino board.
The vehicular Electronic Control Unit (ECU) controls air/fuel injection rate
according to the voltage inputs. 

EcoDrive addresses two main challenges. 


%\textbf{How to model vehicle dynamics and build AFR profile by using OBD parameters, 
%given various vehicle types, transmission types and road conditions?}
\textbf{a) Model Vehicle Dynamics based on OBD Parameters}.
EcoDrive uses the OBD parameters delivered by
the sensing component to build an AFR profile, 
which records instant fuel consumptions of various accelerations under different speeds. 
To this end, EcoDrive models various vehicle forces, 
including propulsion, drivetrain loss, wind resistance
and grade resistance, as functions of instant fuel consumption.  
First, we model propulsion (or output torque) as a function of engine
torque and gear ratio \cite{vong2006prediction, giannelli2005heavy}. 
We use AFR to model engine torque, and use the ratio between RPM
and vehicular speed to model gear ratio. 
Second, we represent drivetrain loss and wind resistance as a function
of vehicular speed \cite{andersson2012online}.
The coefficients are estimated from recorded data traces where
the car is driving at constant speeds. 
The basic idea is that the sum of resistances is equal to
propulsion when the car is driving at a steady-state speed. 
Third, we model grade resistance by altitude changes over road segments. 
The altitudes of the locations are obtained from 
National Elevation Dataset \cite{nationalelevation}.   
Based on the three models, AFR is modeled as a function of 
speed, acceleration and road conditions. 


\textbf{b) Control Air/Fuel Rate and Vehicular Speed to Improve Fuel Efficiency}.
%\textbf{How to control accelerations and cruising speeds to improve
%gas mileage, given various road segment lengths and speed limits?}
EcoDrive controls vehicular speed by emulating gas pedal. 
It sends the emulated gas pedal position values to the
Arduino board which then converts the position
values to corresponding output voltages.
The output voltages are delivered to the ECU through a 6 Pin Connector.  
The problem is how to adjust vehicular speeds to travel through
a certain distance with the lowest fuel consumption.
We solve this problem by using dynamic programming. 
Each state of the dynamic programming model records 
the minimum air/fuel cost that allows the car to achieve
the current speed at the current location. 
In this model, speed can only increase and the last state of each
speed records the minimum fuel consumption if the car reaches the pre-assigned distance at that speed. 
We call the speed with minimum fuel consumption the target speed.  
By backtracking the state matrix from the last state
of target speed, EcoDrive can obtain the desired AFR at each speed. 
EcoDrive adjusts air/fuel injection rate based on real-time sensed vehicular speed. 
Once the vehicular speed reaches the target speed, EcoDrive
enters a cruising state and commands a constant air/fuel injection rate
until the car reaches the pre-assigned distance. 



\textbf{EcoDrive Prototype}. We build a prototype of EcoDrive in an off-the-shelf mobile embedded platform. 
The prototype is installed and tested on a 2011 Chevrolet Impala. 
We test EcoDrive on the Impala for more than 100 miles in both urban and highway environments.
We evaluate the fuel consumption of EcoDrive and human drivers on different road segments.   
In urban areas, EcoDrive achieves 10\%-40\% higher fuel efficiency than four recruited human drivers.
On highway, we evaluate EcoDrive on two highway segments with different target speeds.    
In our user tests, EcoDrive has over 30\% improvements compared to different human drivers. 
In comparison with cruise control, which is more fuel efficient
than human drivers in the traces we collected, 
EcoDrive achieves an average of 10\% higher fuel efficiency.  
We evaluate the performance of EcoDrive on other vehicles by using trace-driven simulation
based on the 10,000 miles data collected from 12 different vehicles. 
We further find that instant
fuel economy display on regular vehicles is misleading and cruise
control is fuel consuming during speed changes (either requested
by user or affected by road conditions).  



\textbf{Contributions}. 
EcoDrive is an independent fuel consumption sensing and control system
that can improve fuel efficiency. 
The system is implemented on an embedded platform 
that can be easily installed on regular vehicles. 
We model various vehicle forces as functions of instant fuel consumption
and the models are evaluated by utilizing 
10,000 miles of driving traces collected from 12 vehicles. 
EcoDrive is installed on a regular vehicle and evaluated by more than 100 miles of driving in both urban and highway environments, 
and demonstrated to improve fuel efficiency compared to cruise control system
of the vehicle and to human drivers.  






\section{RTDrive}


A self-driving vehicle is one that is capable of sensing 
its environment and navigating itself without human input \cite{wikiselfdrivingcar}. 
It uses a variety of techniques to sense its surroundings,
such as LIDAR, RADAR, odometry, and computer vision. 
It uses these different sensor inputs 
to understand its environment, 
recognize various road conditions, traffic lights, road signs,
lane boundaries, and track surrounding vehicles.
The potential benefits of self-driving vehicles
include increased safety, increased mobility and lower
costs. 
It is estimated that self-driving vehicles can reduce 90\%
of the accidents and prevent up to \$190 
billion in damages and health-costs annually
\cite{litman2014autonomous}.


Many commercial and academic endeavors are putting significant resources
for the development and tests
of such self-driving systems \cite{waymo, benz, autox}.
For example, Google started its self-driving project in 2009,  
and has spent more than \$1 billion
in building and testing fully self-driving vehicles~\cite{googlespend}. 
While legal and political challenges remain in its widespread adoption,
there are also some technical bottlenecks on the way of developing
completely reliable self-driving systems.


All self-driving systems make
decisions based on the perception of the environment and
predefined traffic rules. However, there has been occasional
failures of these systems when they have encountered scenarios
that were hitherto unseen. For instance, based on the situation
 in a construction
zone, human drivers would realize that it is permissible to cross
over a double yellow line by following the appropriately placed
cones  (which otherwise is illegal to cross
in the US), while a self-driving vehicle may not be able to
do so, and therefore be unable to move forward. Similarly in
poor weather conditions or due to traffic light malfunctions,  
the cues from different sensors may
contradict each other leading to confusion in decision making.

In general, the road rules are complex and may conflict with
each other, i.e., the system has to understand when
to follow cones and ignore lane markers, 
and when to obey a road worker and disobey traffic
signs.
%To address the last mile challenge, some recent work \cite{kang2018rc} 
%propose to use specially designated remote human operators
%to augment self-driving system when it fails to 
%perceive or handle current situations. 
While it is tempting to return control (during the failure of the self-driving
function) to a local human driver situated in the vehicle, 
it is foreseeable a future of driverless cabs carrying only
underage or licenseless passengers.
Hence, we expect that remote drivers can multiplex and manage
a large group of vehicles making scalability feasible.
However, accomplishing remote driving for a vehicle requires careful tuning 
of (wireless-based) network parameters, media content and their formats,
and control experience with some real-time constraints between the vehicle 
and the remote driving station.

%It is well established that human
%drivers, especially experts, are capable of making good judgement calls
%in face of contradictory or inadequate inputs, that sometimes limit 
%a learning system that has yet to encounter a scenario before.
%While it is tempting to return control (during the failure of the self-driving
%function) to a local human driver situated in the vehicle, 
%it is foreseeable a future of driverless cabs carrying only
%underage or licenseless passengers.
%Hence, we expect that remote drivers can multiplex and manage
%a large group of vehicles making scalability feasible.



We present RTDrive, a remote driving framework
that augments self-driving system when it fails to 
percept and/or handle current situations. 
RTDrive consists of a live streaming system
and a remote control server. 
The live streaming system can encode videos by using
a context-aware video encoding algorithm. 
It also includes a live streaming protocol that 
carries out a consistent-latency
view mechanism to make the view of the operator
more smooth. 
The framework also consists of several modules, video codec, 
Forward Error Correction (FEC), vehicle dynamics sensing,
lane boundary detection, object detection etc.,
that enable further extension, optimization and innovation.  



\textbf{Context-Aware Video Encoding}.
The context-aware video encoding algorithm can 
sense vehicle dynamics and based on which it 
selects the optimal video encoding bitrate and key frame intervals. 
In video live streaming, the video is encoded into two
frames, I-frame and P-frame.
I-frame is also called key frame and it can be 
decoded into a complete image. 
P-frame only encodes the difference between current
frame to previous I-frames and P-frames. 
The context-aware video encoding algorithm
can dynamically adjust the number of I-frames
and P-frames to improve video encoding efficiency. 
The intuition is to adjust the key frame interval 
based on the frequency of camera view changes. 
For example, in the cases of turning or high speed scenarios,
there should be more I-frames since P-frames cannot
carry enough information which may lead to severe quality degradation. 
Through real world trip data collection and replay,
the context-aware video encoding algorithm can outperform
the default encoding algorithm by 10\%-30\% in various trips. 

\textbf{Live Streaming Protocol}.
We design and implement a live streaming protocol. 
It consists of several modules such as UDP/TCP, FEC, 
bandwidth and packet loss rate estimation. 
We discuss the design choices and evaluate these modules
under various network parameter settings. 
Also, a consistent-latency view algorithm 
is designed to deliver smooth
videos to improve the remote control experience. 
It uses a buffer to order the frames based on the timestamps
and deliver to the frame display engine only when it is 
the its order.
It achieves this goal by tracking two parameters, 
the latency difference and the latency deviation.   
The latency difference between
the live streaming system and the server is
tracked by using a low pass filter. 
The deviation of the latency difference is also recorded. 
Through a user study with 20 participates on controlling the Android-powered
vehicle, the operators with consistent-latency have
2x better control precision. 


This paper makes the following contributions:
\begin{itemize}
\setlength\itemsep{0em}

\item We design and implement RTDrive, a live streaming and remote control
framework, that can be used to view video stream and
control the vehicle remotely in real time. 
RTDrive can augment self-driving systems when they are
failed to percept the environment under various unpredictable conditions. 

\item We present a context-aware video encoding algorithm,
which can encode video frames according to vehicle dynamics. 
According to the traces we collected, it is able to improve
the video encoding efficiency by 10\% to 30\% on average in 
various driving scenarios. 

\item We propose a consistent-latency live streaming protocol,
which can adapt to wireless network conditions and buffer frames
for smooth display. 
It includes various modules such as bandwidth estimation, loss rate 
estimation, FEC encoding and frame buffering. 
According to a user study with 20 participates, 
the consistent-latency view algorithm improve the control
precision by more than 2x on average in a parking
task. 
\end{itemize}





\section{Contributions}



\section{Outline}

The rest of the thesis is organized as follows. 
In Chapter \ref{chapter_drivesense}, we present
our driving analytics system DriveSense, which leverages smartphone
sensors to monitor driving behaviors.
In Chapter \ref{chapter_ecodrive}, we present EcoDrive, an in-vehicle
system that can . 
In Chapter \ref{chapter_rtdrive}, we propose the live streaming
and remote control framework RTDrive, which can augment
self-driving systems upon occational failures. 
In Chapter \ref{chapter_relatedwork}, we compare our work with prior
approaches and systems to monitor, assist or even replace human
drivers. 
We conclude and discuss the avenues for further research in Chapter \ref{chapter_conclusion}.



