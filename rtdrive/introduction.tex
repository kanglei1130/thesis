
A self-driving vehicle is one that is capable of sensing 
its environment and navigating itself without human input \cite{wikiselfdrivingcar}. 
It uses a variety of techniques to sense its surroundings,
such as LIDAR, RADAR, odometry, and computer vision. 
It uses these different sensor inputs 
to understand its environment, 
recognize various road conditions, traffic lights, road signs,
lane boundaries, and track surrounding vehicles.
The potential benefits of self-driving vehicles
include increased safety, increased mobility and lower
costs. 
It is estimated that self-driving vehicles can reduce 90\%
of the accidents and prevent up to \$190 
billion in damages and health-costs annually
\cite{litman2014autonomous}.


Many commercial and academic endeavors are putting significant resources
for the development and tests
of such self-driving systems \cite{waymo, benz, autox}.
For example, Google started its self-driving project in 2009,  
and has spent more than \$1 billion
in building and testing fully self-driving vehicles~\cite{googlespend}. 
While legal and political challenges remain in its widespread adoption,
there are also some technical bottlenecks on the way of developing
completely reliable self-driving systems.


All self-driving systems make
decisions based on the perception of the environment and
predefined traffic rules. However, there has been occasional
failures of these systems when they have encountered scenarios
that were hitherto unseen. For instance, based on the situation
 in a construction
zone, human drivers would realize that it is permissible to cross
over a double yellow line by following the appropriately placed
cones  (which otherwise is illegal to cross
in the US), while a self-driving vehicle may not be able to
do so, and therefore be unable to move forward. Similarly in
poor weather conditions or due to traffic light malfunctions,  
the cues from different sensors may
contradict each other leading to confusion in decision making.

In general, the road rules are complex and may conflict with
each other, i.e., the system has to understand when
to follow cones and ignore lane markers, 
and when to obey a road worker and disobey traffic
signs.
%To address the last mile challenge, some recent work \cite{kang2018rc} 
%propose to use specially designated remote human operators
%to augment self-driving system when it fails to 
%perceive or handle current situations. 
While it is tempting to return control (during the failure of the self-driving
function) to a local human driver situated in the vehicle, 
it is foreseeable a future of driverless cabs carrying only
underage or licenseless passengers.
Hence, we expect that remote drivers can multiplex and manage
a large group of vehicles making scalability feasible.
However, accomplishing remote driving for a vehicle requires careful tuning 
of (wireless-based) network parameters, media content and their formats,
and control experience with some real-time constraints between the vehicle 
and the remote driving station.

%It is well established that human
%drivers, especially experts, are capable of making good judgement calls
%in face of contradictory or inadequate inputs, that sometimes limit 
%a learning system that has yet to encounter a scenario before.
%While it is tempting to return control (during the failure of the self-driving
%function) to a local human driver situated in the vehicle, 
%it is foreseeable a future of driverless cabs carrying only
%underage or licenseless passengers.
%Hence, we expect that remote drivers can multiplex and manage
%a large group of vehicles making scalability feasible.



In this paper, we present RTDrive, a remote driving framework
that augments self-driving system when it fails to 
percept and/or handle current situations. 
RTDrive consists of a live streaming system
and a remote control server. 
The live streaming system can encode videos by using
a context-aware video encoding algorithm. 
It also includes a live streaming protocol that 
carries out a consistent-latency
view mechanism to make the view of the operator
more smooth. 
The framework also consists of several modules, video codec, 
Forward Error Correction (FEC), vehicle dynamics sensing,
lane boundary detection, object detection etc.,
that enable further extension, optimization and innovation.  



\textbf{Context-Aware Video Encoding}.
The context-aware video encoding algorithm can 
sense vehicle dynamics and based on which it 
selects the optimal video encoding bitrate and key frame intervals. 
In video live streaming, the video is encoded into two
frames, I-frame and P-frame.
I-frame is also called key frame and it can be 
decoded into a complete image. 
P-frame only encodes the difference between current
frame to previous I-frames and P-frames. 
The context-aware video encoding algorithm
can dynamically adjust the number of I-frames
and P-frames to improve video encoding efficiency. 
The intuition is to adjust the key frame interval 
based on the frequency of camera view changes. 
For example, in the cases of turning or high speed scenarios,
there should be more I-frames since P-frames cannot
carry enough information which may lead to severe quality degradation. 
Through real world trip data collection and replay,
the context-aware video encoding algorithm can outperform
the default encoding algorithm by 10\%-30\% in various trips. 

\textbf{Live Streaming Protocol}.
We design and implement a live streaming protocol. 
It consists of several modules such as UDP/TCP, FEC, 
bandwidth and packet loss rate estimation. 
We discuss the design choices and evaluate these modules
under various network parameter settings. 
Also, a consistent-latency view algorithm 
is designed to deliver smooth
videos to improve the remote control experience. 
It uses a buffer to order the frames based on the timestamps
and deliver to the frame display engine only when it is 
the its order.
It achieves this goal by tracking two parameters, 
the latency difference and the latency deviation.   
The latency difference between
the live streaming system and the server is
tracked by using a low pass filter. 
The deviation of the latency difference is also recorded. 
Through a user study with 20 participates on controlling the Android-powered
vehicle, the operators with consistent-latency have
2x better control precision. 


This paper makes the following contributions:
\begin{itemize}
\setlength\itemsep{0em}

\item We design and implement RTDrive, a live streaming and remote control
framework, that can be used to view video stream and
control the vehicle remotely in real time. 
RTDrive can augment self-driving systems when they are
failed to percept the environment under various unpredictable conditions. 

\item We present a context-aware video encoding algorithm,
which can encode video frames according to vehicle dynamics. 
According to the traces we collected, it is able to improve
the video encoding efficiency by 10\% to 30\% on average in 
various driving scenarios. 

\item We propose a consistent-latency live streaming protocol,
which can adapt to wireless network conditions and buffer frames
for smooth display. 
It includes various modules such as bandwidth estimation, loss rate 
estimation, FEC encoding and frame buffering. 
According to a user study with 20 participates, 
the consistent-latency view algorithm improve the control
precision by more than 2x on average in a parking
task. 
\end{itemize}



