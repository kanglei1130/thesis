
\subsection{Live Video Streaming}
\nop{
Internet video streaming protocols can be classified into
two major fields based on application scenarios.
One set of protocols is streaming pre-recorded video 
over Internet, such as YouTube and Netflix \cite{rao2011network}.
Another set of protocols is streaming live video, 
which are also known
as Voice over Internet Protocol (VoIP), 
such as Skype and Google Hangouts 
\cite{yu2014can, xu2012video, zhang2012profiling}.
Streaming recorded video meets with TCP-like protocols in their nature, 
i.e., video streaming adopts pre-fetching and buffering 
to achieve smooth play-out.
Hulu \cite{adhikari2015measurement} uses 
encrypted RTMP (Real Time Messaging Protocol).
Youtube is using Real Time Streaming Protocol (RTSP) 
and Netflix is using TCP \cite{rao2011network}
}

Measuring live streaming applications such as Skype and Hangout
is an active area of research. 
Streaming live video is sensitive to network latency,
so such applications use UDP-like protocols by default
and switch to TCP if UDP is 
blocked \cite{yu2014can, xu2012video, zhang2012profiling}.
\cite{li2017measurement} measures the performance of Skype 
over today's LTE networks and illustrate the inefficiencies of Skype protocols.
\cite{yu2014can} measures video quality of different 
VoIP applications over wireless and mobile networks. 
\cite{xu2012video} measures the design choices and 
performance of Skype, iChat and Google+ 
under a wide range of ``best-effort'' network conditions.
\cite{zhang2012profiling} measures how Skype adjusts its sending rate, FEC redundancy,
video rate and frame rate under various packet loss rate, 
propagation delay and available network bandwidth.
Our live streaming framework has similar network protocol
requirements with VoIP application, 
the implementation and context-aware video encoding algorithm
differentiate us from the measurement study of exisiting
VoIP services. 
Real time streaming of pre-recorded video such as YouTube and Netflix \cite{rao2011network}
are not closely related to the application scenario in this work. 


\subsection{Sensing Vehicle Dynamics}

The smartphone built-in sensors enable lots of vehicular applications
that sense vehicle movements and capture driving behaviors 
\cite{hansenspeed, wang2013sensing, chen2015invisible, uber, cmtelematics}. 
\cite{hansenspeed} uses an accelerometer to estimate vehicular speed. 
\cite{wang2013sensing} captures turns to determine driver phone
use by comparing centrifugal force with a reference point. 
\cite{chen2015invisible} develops a middleware that can detect 
and differentiate various vehicle maneuvers, 
including lane changes, turns, and driving on curvy roads,
by using non-vision sensors.
\cite{singh2013using, fazeen2012safe} use inertial sensors to detect the driving quality of the
driver.
Our work utilize these techniques to sense driving context 
to improve video encoding efficiency.  

 
\subsection{Self-Driving Systems}

There are many corporations and researchers are developing fully self-driving
techniques, such as Waymo, Mercedes-Benz and AutoX \cite{waymo, benz, autox}.
Waymo uses LIDAR as the primary input for object detection \cite{waymo}. 
AutoX proposes camera-first self-driving solution to reduce
the cost to build a self-driving vehicle \cite{autox}.  
\cite{cvpr17chen} presents a sensory-fusion perception framework 
that combines LIDAR point cloud and RGB images as input and 
predicts oriented 3D bounding boxes. 
\cite{leonard2008perception} describes the architecture and implementation 
of an autonomous vehicle designed to navigate using locally perceived 
information in preference to potentially inaccurate or incomplete map data. 
\cite{lee2016internet} presents networked self-driving vehicles to coordinate 
and form an edge computing platform. 
We believe our remote control framework can act the safe backup
for such self-driving systems.
Different from \cite{kang2018rc}, which presents high level possible 
challenges and directions, 
our work present the design and implementation of a working system.  


\subsection{Reducing Network Latency}

Reducing network latency is an active area of research. 
\cite{singla2014internet} investigates the causes of latency inflation in the
Internet and proposes a grand challenge for the networking research 
community: a speed-of-light Internet. 
\cite{zhang20145g, agiwal2016next} propose various architectures and techniques
for high capacity and low latency 5th generation mobile networks. 
\cite{huang2013depth} develops a passive measurement tool to 
study the inefficiency in today's LTE networks. 
\cite{ali2013quality} presents the features to improve
quality of service in LTE networks. 
\cite{stonebraker20058} discusses the requirements of system
design for real-time streaming. 
\cite{song2017wifi} presents a Wi-Fi based roadside hotspot network to
operate at vehicular speeds with meter-sized picocells.
\cite{lee2015outatime} uses speculation to predict future frames to reduce 
latency for mobile cloud gaming.
\cite{tu2016volte} presents the inefficiencies of current VoLTE architectures. 
All these work can inspire the design of remote control systems for self-driving vehicles.
\cite{singla2014internet} reveals that
infrastructural inefficiencies and protocol
overheads cause today's Internet latency. 




