
\chapter{Related Work}
\label{chapter_relatedwork}

\section{Economic Driving}

\subsection{Cruise Control}

%\textbf{Cruise Control}. 
Cruise control \cite{cruise_control} is a system that automatically
controls the vehicle to drive in a certain speed. 
\cite{bengtsson2001adaptive} introduces Adaptive Cruise Control (ACC)
that adjusts vehicle speed based on distance to cars ahead
by using radar. 
\cite{lin2009effects} studies the effects of time-gap
settings of adaptive cruise control on driving performance
in a bus driving simulator. 
\cite{loos2011adaptive} studies a distributed car control system
that every car is controlled by adaptive cruise control. 
\cite{ioannou1993autonomous} develops an intelligent cruise control system 
for automatic vehicles and verify the effectiveness of the system
under emergency situations by simulation.  
EcoDrive is more fuel efficient than adaptive cruise control 
in two ways. 
First, EcoDrive is able to improve fuel efficiency by 
adjusting speed according to road conditions and vehicle types. 
Second, EcoDrive tracks the fuel efficiency under different cruising
speeds and provides a tradeoff between fuel efficiency and travel time. 

\subsection{Fuel Efficiency Enhancement}

%\textbf{Fuel Efficiency Enhancement}. 
Increasing vehicular fuel efficiency has been a topic 
of much recent research \cite{ganti2010greengps, koukoumidis2011signalguru, 
boriboonsomsin2010eco, tulusan2012providing}.
GreenGPS \cite{ganti2010greengps} requires a fuel map server
to record the statistical route fuel usage and calculate the most
fuel efficient route for drivers. 
SignalGuru \cite{koukoumidis2011signalguru} requires vehicles sharing
traffic light information so that drivers can adjust the speed of
the vehicle to avoid stops in front of traffic lights. 
\cite{boriboonsomsin2010eco} uses an on-board eco-driving device 
that provides instantaneous fuel economy
feedback affects driving behaviors, and consequently fuel economy. 
\cite{tulusan2012providing} evaluates the impacts of a smartphone application
on driving behaviors to enhance fuel efficiency. 
Different from existing approaches that are focusing on advising human drivers to adjust driving behaviors, 
EcoDrive is designed to assist human drivers.
Both \cite{lang2013opportunities} and \cite{lang2014prediction} show that some reduction of fuel consumption can be achieved
by vehicle-to-vehicle communication. 

\subsection{Vehicle Force Modeling}

%\textbf{Vehicle Dynamics Modeling}. 
\cite{andersson2012online} models rolling resistance and air drag based on some measurements 
on the vehicle, e.g., effective areas to model air drag. 
\cite{zanasi2001dynamic} models vehicle transmission system based on gear sizes. 
\cite{canudas1999dynamic} models tire friction based on tire parameters, e.g., radius. 
\cite{brusamarello2010dynamic} presents an electronic system attached on wheel to measure and
monitor of the torque transmitted to the wheel of a moving vehicle.
\cite{vong2006prediction} models engine torque and horsepower by RPM when the
engine is not connected with transmission. 
Our model uses the parameters available on car CAN bus to find the relations between fuel consumption and vehicle speed changes.
Different from existing work, 
our model does not rely on the detail physics properties of the car, 
e.g., tire radius and transmission gear size etc.




\section{Vehicle Dynamics Sensing}

\subsection{Coordinate Alignment and Estimation}

Estimating smartphone orientation is critical 
for object tracking and has also been an active research area. 
The related efforts can be categorized into two main approaches:
absolute coordinates estimation and relative coordinates estimation.
Absolute coordinates estimation is usually used for 
walking direction estimation \cite{shen2013walkie, zhou2014use} etc. 
It is a challenging task due to the limited precision
and accumulated error of IMU sensors \cite{zhou2014use}. 
Using smartphone sensors to track vehicle motions 
is more challenging due to an extra relative orientation
alignment step, i.e., the alignment from smartphone coordinates
to the car's coordinates.
Both steps may introduce errors into the estimation. 
The existing alignment algorithms require
extra inputs, i.e., GPS assistance \cite{Mohan2008Nericell},
offline training \cite{yang2015low}, 
or focus on a one-dimensional motion like lane change and turn 
\cite{wang2013sensing, chen2015invisible}.
However, none of the existing work addressed the challenges caused
by road slopes and misalignment. 


\subsection{Capturing Driving Behaviors}

Capturing driving behaviors by smartphones are of interest
to both academia \cite{wang2013sensing, chen2015invisible} 
and industry \cite{uber, cmtelematics}. 
\cite{wang2013sensing} captures turns to determine driver phone
use by comparing centrifugal force with a reference point. 
\cite{chen2015invisible} develops a middleware that can detect 
and differentiate various vehicle maneuvers, 
including lane changes, turns, and driving on curvy roads,
by using non-vision sensors.
They capture steering events utilizing the z-axis of the gyroscope.
These work rely on the gyroscope and are not 
discussing the problem caused by gravitational force. 
Our software module can accurately estimate vehicle motion
parameters, which can be used as inputs to 
capture driving behavior more accurately. 

\subsection{Vehicular Applications of IMU Sensors}

The IMU sensors in smartphones enable lots of applications
including road condition detection \cite{Mohan2008Nericell},
3D road modeling \cite{yang2015low}, vehicular speed estimation \cite{hansenspeed}.
\cite{Mohan2008Nericell} tracks road surface like 
potholes by using accelerometer and GPS.
\cite{yang2015low} builds road 3D models by using IMU sensors on  a smartphone, but it requires the driver to drive 
around a building to calibrate the magnetic sensor. 
\cite{hansenspeed} uses an accelerometer to estimate speed 
where the GPS is inaccurate. 
Different from these work, the focus of this work is 
conducting coordinate alignment efficiently and estimating the linear acceleration of the vehicle.
We believe our module can provide input for vehicular applications 
to achieve better efficiency. 




\subsection{Live Video Streaming}
\nop{
Internet video streaming protocols can be classified into
two major fields based on application scenarios.
One set of protocols is streaming pre-recorded video 
over Internet, such as YouTube and Netflix \cite{rao2011network}.
Another set of protocols is streaming live video, 
which are also known
as Voice over Internet Protocol (VoIP), 
such as Skype and Google Hangouts 
\cite{yu2014can, xu2012video, zhang2012profiling}.
Streaming recorded video meets with TCP-like protocols in their nature, 
i.e., video streaming adopts pre-fetching and buffering 
to achieve smooth play-out.
Hulu \cite{adhikari2015measurement} uses 
encrypted RTMP (Real Time Messaging Protocol).
Youtube is using Real Time Streaming Protocol (RTSP) 
and Netflix is using TCP \cite{rao2011network}
}

Measuring live streaming applications such as Skype and Hangout
is an active area of research. 
Streaming live video is sensitive to network latency,
so such applications use UDP-like protocols by default
and switch to TCP if UDP is 
blocked \cite{yu2014can, xu2012video, zhang2012profiling}.
\cite{li2017measurement} measures the performance of Skype 
over today's LTE networks and illustrate the inefficiencies of Skype protocols.
\cite{yu2014can} measures video quality of different 
VoIP applications over wireless and mobile networks. 
\cite{xu2012video} measures the design choices and 
performance of Skype, iChat and Google+ 
under a wide range of ``best-effort'' network conditions.
\cite{zhang2012profiling} measures how Skype adjusts its sending rate, FEC redundancy,
video rate and frame rate under various packet loss rate, 
propagation delay and available network bandwidth.
Our live streaming framework has similar network protocol
requirements with VoIP application, 
the implementation and context-aware video encoding algorithm
differentiate us from the measurement study of exisiting
VoIP services. 
Real time streaming of pre-recorded video such as YouTube and Netflix \cite{rao2011network}
are not closely related to the application scenario in this work. 


\subsection{Sensing Vehicle Dynamics}

The smartphone built-in sensors enable lots of vehicular applications
that sense vehicle movements and capture driving behaviors 
\cite{hansenspeed, wang2013sensing, chen2015invisible, uber, cmtelematics}. 
\cite{hansenspeed} uses an accelerometer to estimate vehicular speed. 
\cite{wang2013sensing} captures turns to determine driver phone
use by comparing centrifugal force with a reference point. 
\cite{chen2015invisible} develops a middleware that can detect 
and differentiate various vehicle maneuvers, 
including lane changes, turns, and driving on curvy roads,
by using non-vision sensors.
\cite{singh2013using, fazeen2012safe} use inertial sensors to detect the driving quality of the
driver.
Our work utilize these techniques to sense driving context 
to improve video encoding efficiency.  

 
\subsection{Self-Driving Systems}

There are many corporations and researchers are developing fully self-driving
techniques, such as Waymo, Mercedes-Benz and AutoX \cite{waymo, benz, autox}.
Waymo uses LIDAR as the primary input for object detection \cite{waymo}. 
AutoX proposes camera-first self-driving solution to reduce
the cost to build a self-driving vehicle \cite{autox}.  
\cite{cvpr17chen} presents a sensory-fusion perception framework 
that combines LIDAR point cloud and RGB images as input and 
predicts oriented 3D bounding boxes. 
\cite{leonard2008perception} describes the architecture and implementation 
of an autonomous vehicle designed to navigate using locally perceived 
information in preference to potentially inaccurate or incomplete map data. 
\cite{lee2016internet} presents networked self-driving vehicles to coordinate 
and form an edge computing platform. 
We believe our remote control framework can act the safe backup
for such self-driving systems.
Different from \cite{kang2018rc}, which presents high level possible 
challenges and directions, 
our work present the design and implementation of a working system.  


\subsection{Reducing Network Latency}

Reducing network latency is an active area of research. 
\cite{singla2014internet} investigates the causes of latency inflation in the
Internet and proposes a grand challenge for the networking research 
community: a speed-of-light Internet. 
\cite{zhang20145g, agiwal2016next} propose various architectures and techniques
for high capacity and low latency 5th generation mobile networks. 
\cite{huang2013depth} develops a passive measurement tool to 
study the inefficiency in today's LTE networks. 
\cite{ali2013quality} presents the features to improve
quality of service in LTE networks. 
\cite{stonebraker20058} discusses the requirements of system
design for real-time streaming. 
\cite{song2017wifi} presents a Wi-Fi based roadside hotspot network to
operate at vehicular speeds with meter-sized picocells.
\cite{lee2015outatime} uses speculation to predict future frames to reduce 
latency for mobile cloud gaming.
\cite{tu2016volte} presents the inefficiencies of current VoLTE architectures. 
All these work can inspire the design of remote control systems for self-driving vehicles.
\cite{singla2014internet} reveals that
infrastructural inefficiencies and protocol
overheads cause today's Internet latency. 



