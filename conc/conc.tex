\chapter{Conclusion and Future Work}
\label{chapter_conclusion}

Modern vehicles are the ultimate mobile computing platforms. 
They are often on the move at significant speeds and 
are equipped with significant embedded computing systems 
that control and manage different functions including providing various types of 
assistance to its driving function. 
The computational capabilities of automobile systems provide opportunities and challenges 
to manage and control every aspects of a vehicle's drive. 
In this thesis, we present three in-vehicle systems, DriveSense, EcoDrive and RTDrive,
which can monitor, assist or even replace in-vehicle drivers 
to improve overall driving performance and experience.
We believe our in-vehicle systems can benefit human drivers
in various aspects.

The future vehicles are going to coordinate
with other vehicles and road infrastructures, 
assist or even replace human drivers, 
and rely on low-cost electrical engine. 
Our futher research will focus on these three 
technical trends. 
We will build practical in-vehicle systems to
communicate with other objects, assist
human drivers, and analyze energy consumption of
electric cars, to improve driving experience,
safety and energy efficiency. 


\textbf{Self-driving Systems} can be used to 
assist or even replace human drivers \cite{googledriverlesscar}. 
We have seen breakthroughs during past decade on driverless
cars. 
But there is still long way to go to make a car to fully handle
the various and complex urban environments.  
A semi-driverless car can be expected in closer
future that the car can identify the situations 
where the control should be returned to a human driver, 
the car can test itself and learn from human drivers
that how to handle unpredictable conditions, 
and the car can assist human drivers in a way when
human drivers are distracted and/or there is no enough
time to react (i.e., an emergency brake system).  


\textbf{Energy Monitoring Systems for Electric Cars} 
can be used to monitor the energy consumtion of electric
cars.
We are collaborating with Innova EV and have four electric
cars on site \cite{electriccaruw} to collect data
from these cars and perform sustainability study. 
In urban environments, the driving is dominated 
by frequent acceleration and braking. 
We have investigated how different acceleration
patterns affect fuel consumption \cite{kang2015ecodrive},
but it is still an open question that the efficiency of
regenerative braking systems and how different braking
behaviors interact with them \cite{ahn2009analysis}. 
We plan to study energy consumption and human driving
behaviors in urban environments, especially
focus on regenerative braking systems. 


\textbf{Vehicle Communciation Systems} can be used to 
coordinate different vechiles to avoid collision and 
coordinate with road infrastructures (e.g., traffic light)
to improve efficiency.
Safety-related communication systems for connected vehicle technology
will likely be based on dedicated 
short-range communications. 
This kind of delay sensitive communication requires
the wireless communication to be fast, secure, and reliable. 
Existing communication protocols like WiFi are not able
to meet this requirements. 
New delay sensitive protocols are expected to be designed to 
address the challenges for safety related applications. 
Connected vehicles can also use wireless communication to
communicate with traffic signals, work zones, toll booths, school zones,
and other types of infrastructure. 
There are numerous challenges. 
First, how to track the movement of vehicles
and instantly know the relative and/or absolute
location of the driving vehicle. 
Second, how to identify a particular vehicle
and associated trajectories based on received
wireless signals to avoid collision. 
Third, how to communicate with dedicated traffic
light and estimate the distance to traffic
light or front car. 




