% prelude.tex
%   - titlepage
%   - dedication
%   - acknowledgments
%   - table of contents, list of tables and list of figures
%   - nomenclature
%   - abstract
%============================================================================


\clearpage\pagenumbering{roman}  % This makes the page numbers Roman (i, ii, etc)



% TITLE PAGE
%   - define \title{} \author{} \date{}
\title{Improving Datacenter Network Performance via the Intelligent Network Edge}
\author{Lei Kang}
\date{}
%   - The default degree is ``Doctor of Philosophy''
%     (unless the document style msthesis is specified
%      and then the default degree is ``Master of Science'')
%     Degree can be changed using the command \degree{}
\degree{Doctor of Philosophy}
%   - The default is dissertation, unless the document style
%     msthesis was specified in which case it becomes thesis.
%     If msthesis is specified for the MS margins, you can
%     still have a dissertation if you specify \disseration
%\disseration
%   - for a masters project report, specify \project
%\project
%   - for a preliminary report, specify \prelim
%\prelim
%   - for a masters thesis, specify \thesis
\thesis
%   - The default department is ``Electrical Engineering''
%     The department can be changed using the command \department{}
\department{Computer Sciences}
%   - once the above are defined, use \maketitle to generate the titlepage
\oralexamdate{5/x/2018}
\committeeone{Suman Banerjee, Professor, Computer Science}
\committeetwo{Mohit Gupta, Assistant Professor, Computer Science}
\committeethree{Soyoung Ahn, Associate Professor, Civil and Environmental Engineering}
\committeefour{TBD, Professor, Computer Science}
\committeefive{TBD, Professor, Electrical and Computer Engineering}
\date{2018}


\maketitle

% COPYRIGHT PAGE
%   - To include a copyright page use \copyrightpage
\copyrightpage

% DEDICATION
\begin{dedication}
\emph{To my loved ones}
\end{dedication}

% ACKNOWLEDGMENTS
\begin{acknowledgments}

It has been a long but rewarding journal at UW-Madison.

First of all, I would like to thank my Ph.D. advisor, Prof. Suman Banerjee. 
He has done an amazing job in guiding my graduate studies, transforming me
from a college graduate without knowing what to do, to a confident Ph.D. with
solid skills at hands. 
He changed me from a paper driven researcher to a problem solving researcher. 
I deeply appreciate the time, patience, care, 
and support Prof. Banerjee has devoted to guiding my professional
development, along with the help on many personal matters. 
 
I would like to thank the members of my thesis committee, TBD. 
Their valuable suggestions and comments greatly improved this thesis.
 
I would like to thank my MPhil advisor, Prof. Lionel Ni. 
He is a decent and kind man. 
He always take care of his students, in every way that is possible, 
while he is very busy with academic and administrative affairs. 


I would like to thank my internship hosts, Bozidar Radunovic, 
Shizhen Zhao, Yadi Ma and Junlan Zhou. 
I would like to thank my other research collaborators, 
Peng Liu,
Bozhao Qi,
Wei Zhao,
Dan Janecek,
Yijing Zeng,
Haoran Qiu,
Allen Liu
Working with these brilliant people is a great gift to me. 

I am also quite fortunate to have great friends that I can hang out with and have
beer together: 
Keqiang He,
Suli Yang,
Jianqiao Zhu,
Yuqi He,
Shiwei Zhou,
Na Li,
Ze Wang,
Xuan Wang,
Yuzhang Zang,
Xue Yin,
Mengguo Jing,
Zhen Di,
Zhe Yang,
Chenxiao Guan,
Xun Zhao,
Ziqi Yang,
Ming Gao,
Yan Zhai,
Liang Wang,
Xiujun Li,


Last but definitely the most important one, my wife.
I still remember when we first met 5 years ago in a classroom. 
The first sentence my wife (Xuan) said to me is to ask my phone number, 
which is truly surprising, as there were no girls initiatively asked my phone number before. 
An even more surprising thing, which I found years later, is she asks the 
phone numbers of almost every Chinese students have the same class with her 
so that she can discuss homework with them.
She makes me a better person and I can hardly imagine
a life without her. 


\end{acknowledgments}

% CONTENTS, TABLES, FIGURES
\tableofcontents
%%\listoftables
%%\listoffigures



\advisorname{Suman Banerjee}
\advisortitle{Professor}



\begin{abstract}



Vehicles bring us travel convenience as well as many environment and safety issues. 
In recent decades, many in-vehicle systems are designed to improve driving performance 
and experience by utilizing the computing power of modern vehicles,
such as cruise control, driver distraction monitoring,
and self-driving systems.   
In this thesis, we present three systems fall in this domain, 
called DriveSense, EcoDrive and RTDrive, 
that are able to monitor, assist or even
replace human drivers to achieve better driving performance and experience. 

DriveSense is a smartphone vehicular application that 
senses vehicle dynamics by built-in sensors and provide feedback to drivers 
on aggressive events to improve their driving safety awareness.
Sensing vehicle dynamics by smartphone sensors require coordinate alignment
between the smartphone and the car. 
We found that even gentle road slopes may cause 
severe coordinate misalignment and acceleration over/under estimation. 
To resolve these problems, we propose slope-aware
coordinate alignment algorithm and linear acceleration
estimation method to reduce alignment training time and improve
linear acceleration estimation accuracy.  


EcoDrive is an in-vehicle system that conducts fuel consumption sensing and control 
to improve fuel efficiency and reduce carbon emissions. 
EcoDrive models instant fuel consumption based on vehicle 
parameters collected from On-board diagnostics (OBD) port. 
According to the model, it controls the gas pedal position sensor
to adjust fuel injection rate according to road segment distance 
and speed limit. 
By using careful control of fuel injection rate, it is able
to improve fuel efficiency comparing to human drivers. 


RTDrive is a live streaming and remote control system to 
handle self-driving system failures. 
A human operator can control the vehicle remotely, 
only when self-driving system failures occur, 
which may be due to bad weather, malfunction,
contradiction in sensory inputs, and other such conditions.
RTDrive consists of a context-aware video encoding method
and a live streaming protocol.
The context-aware video encoding method can improve video
streaming quality by adjusting encoding parameters according to vehicle dynamics.   
We also implement a consistent-latency view mechanism to
smooth the video frames, 
under which the remote driver can have more precise control
over the vehicle. 


\end{abstract}


\clearpage\pagenumbering{arabic} % This makes the page numbers Arabic (1, 2, etc)
