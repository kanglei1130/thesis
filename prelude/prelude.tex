% prelude.tex
%   - titlepage
%   - dedication
%   - acknowledgments
%   - table of contents, list of tables and list of figures
%   - nomenclature
%   - abstract
%============================================================================


\clearpage\pagenumbering{roman}  % This makes the page numbers Roman (i, ii, etc)



% TITLE PAGE
%   - define \title{} \author{} \date{}
\title{Improving Datacenter Network Performance via the Intelligent Network Edge}
\author{Lei Kang}
\date{}
%   - The default degree is ``Doctor of Philosophy''
%     (unless the document style msthesis is specified
%      and then the default degree is ``Master of Science'')
%     Degree can be changed using the command \degree{}
\degree{Doctor of Philosophy}
%   - The default is dissertation, unless the document style
%     msthesis was specified in which case it becomes thesis.
%     If msthesis is specified for the MS margins, you can
%     still have a dissertation if you specify \disseration
%\disseration
%   - for a masters project report, specify \project
%\project
%   - for a preliminary report, specify \prelim
%\prelim
%   - for a masters thesis, specify \thesis
\thesis
%   - The default department is ``Electrical Engineering''
%     The department can be changed using the command \department{}
\department{Computer Sciences}
%   - once the above are defined, use \maketitle to generate the titlepage
\oralexamdate{5/1/2018}
\committeeone{Suman Banerjee, Professor, Computer Science}
\committeetwo{Srinivasa A. Akella, Professor, Computer Science}
\committeethree{Eric J. Rozner, Research Staff Member, IBM Research}
\committeefour{Michael M. Swift, Associate Professor, Computer Science}
\committeefive{Xinyu Zhang, Assistant Professor, Electrical and Computer Engineering}
\date{2018}


\maketitle

% COPYRIGHT PAGE
%   - To include a copyright page use \copyrightpage
\copyrightpage

% DEDICATION
\begin{dedication}
\emph{To my loved ones}
\end{dedication}

% ACKNOWLEDGMENTS
\begin{acknowledgments}

It has been a long but rewarding journal at UW-Madison.

First of all, I would like to thank my Ph.D. advisor, Prof. Suman Banerjee. 
He has done an amazing job in guiding my graduate studies, transforming me
from a college graduate without knowing what to do, to a confident Ph.D. with
solid skills at hands. 
He changed me from a paper driven researcher to a problem solving researcher. 
I deeply appreciate the time, patience, care, 
and support Prof. Banerjee has devoted to guiding my professional
development, along with the help on many personal matters. 
 
I would like to thank the members of my thesis committee, TBD. 
Their valuable suggestions and comments greatly improved this thesis.
 
I would like to thank my MPhil advisor, Prof. Lionel Ni. 
He is a decent and kind man. 
He always take care of his students, in every way that is possible, 
while he is very busy with academic and administrative affairs. 


I would like to thank my internship mentors, Bozidar

 
I would like to thank my research collaborators, 
Peng Liu,
Bozhao Qi,
Wei Zhao,
Dan Janecek,
. 
Working with these brilliant people is a great gift to me. 
I am also quite fortunate to have great friends that I can hang out with and have
beer together: 
Keqiang He,
Suli Yang,


Last but definitely the most important one, my wife.  


\end{acknowledgments}

% CONTENTS, TABLES, FIGURES
\tableofcontents
%%\listoftables
%%\listoffigures



\advisorname{Suman Banerjee}
\advisortitle{Professor}



\begin{abstract}



Vehicles bring us travel convenience as well as many problems 
such as environment pollution, road congestion and fatalities. 
We propose building smart in-vehicle systems and smartphone enabled
vehicular applications to provide various types of 
assistance to remedy these issues. 
Such systems sense and model vehicle dynamics from vehicle parameters and
third party sensors, based on which they provide guidance
and control to assist human drivers and achieve better driving performance. 
EcoDrive is one such system we built that conduct fuel consumption sensing and control 
to improve fuel efficiency and reduce carbon emissions. 
EcoDrive models instant fuel consumption based on vehicle 
parameters collected from On-board diagnostics (OBD) port. 
According to the model, it controls the gas pedal position sensor
to adjust fuel injection rate according to road segment distance 
and speed limit. 
By using careful control of fuel injection rate, it is able
to improve fuel efficiency comparing to human drivers. 
We also implemented a smartphone vehicular application, called DriveSense, 
that senses vehicle dynamics by built-in sensors and provide feedback to drivers 
on aggressive events to improve their driving safety awareness.
Sensing vehicle dynamics by smartphone sensors require coordinate alignment
between the smartphone and the car. 
We found that even gentle road slopes may cause 
severe coordinate misalignment and acceleration over/under estimation. 
To resolve these problems, we propose slope-aware
coordinate alignment algorithm and linear acceleration
estimation method to reduce alignment training time and improve
linear acceleration estimation accuracy.  
Based on our past experiences on in-vehicle systems and driving behavior study, 
we propose two future work to further improve driving safety 
and efficiency by utilizing smartphone sensors. 
First, we propose to evaluate the performance of smartphone GPS and 
IMU sensors for capturing driving behaviors, 
and examine the benefits when combining GPS and IMU sensors.
Second, we propose to combine smartphone cameras, GPS and IMU sensors to detect driver phone use. 

\end{abstract}


\clearpage\pagenumbering{arabic} % This makes the page numbers Arabic (1, 2, etc)
