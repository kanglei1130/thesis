

Smartphone-based sensing has opened up a whole gamut of applications and services
across many domains. In the world of the transportation system, it is being increasingly used
for crowd-sourcing various forms of information, ranging from road and traffic conditions to data on traffic light patterns,
and even interesting annotations by participants.
One such application is the ability to independently monitor driving behavior --- how well is one
driving the vehicle, e.g., aggressive driving actions, such as rough acceleration, hard brakes, sharp turns, and more.
The popularity of smartphones enable innovative ways of monitoring such behaviors and actions.
For example, Cambridge Mobile Telematics \cite{cmt} 
develops a smartphone
app to capture driving behaviors and monitor driver
distractions.
Similarly, the well-known ride-sharing company Uber announced
it will start tracking Uber drivers' driving behaviors
with their smartphones and give them feedbacks that are more detailed than 
the five-star rating customers leave for each driver \cite{uber}.
Another set of applications that can leverage sensing of motion parameters is to understand road conditions, to
say, detect potholes \cite{eriksson2008pothole} or to detect icy stretches during a bad snow event.

A common technique to detect various vehicle parameters, especially acceleration, braking, and turns ---
events which happen in short timescales --- is to use built-in inertial measurement unit (IMU) 
sensors available in the mobile devices \cite{wang2013sensing,hansenspeed,chen2015invisible}. 
The general approach taken by such systems is to measure the three-dimensional acceleration
using the accelerometer and to measure the three-dimensional relative rotation speed using
a gyroscope, including various de-noising techniques to make precise estimates.
An important step in such design is one of coordinate alignment where the system needs to
calculate the rotation matrix that translates 
the coordinates of the smartphone to that of the vehicle. 



However, we find that exsiting approaches may lead to significant estimation
errors (especially acceleration) if 
the smartphone is not tightly mounted, there are relative orientation changes,
and/or the vehicle is moving over road slopes. 
If the mobile device is not tightly fixed
in a mount, the device may move occasionally within the mount and its orientation 
relative to the vehicle may frequently change causing measurement
errors. Worse, if the device is occasionally held in a person's hand, the update process needs
to be applied continuously significantly exacerbating the challenge. Finally,
even if the device is perfectly fixed to a mount, there are potential errors since the device itself
might not be able to determine whether the vehicle itself is on flat earth or is 
tilted due to the slope of a road.





\nop{
As a consequence, it is worth considering whether an alternative approach to detecting these
motion parameters is possible through the use of GPS.
Clearly, GPS is widely used in many outdoor applications to identify location and even speed,
and acceleration can be determined by observing changes in speed.
Use of GPS has one big advantage --- we do not need to do orientation corrections between
device and vehicle and makes the computation both simpler and more robust.
However, low-cost GPS receivers that are available in common mobile devices have other problems.
They do not always have
sufficient resolution to accurately estimate some short timescale events, e.g., rotation
speeds, acceleration and brakes.
Further, such GPS devices do not always have the greatest accuracy and resolution in location
estimation, 
which may also lead to additional errors in estimating acceleration
and brakes.
}


\nop{
In this paper, we therefore, first, conduct a study of whether some of the vehicular motion
parameters of interest, especially those related to acceleration, brakes, and turns, 
can be estimated accurately when there are slopes.
To understand the performance of GPS and inertial sensors for these purposes,
we use measured data from vehicles collected through more than 10,000 miles of
motion, 
 which provide GPS data and inertial sensor measurements from mobile devices and 
 some independently logged speed data from the On-board Diagnostics (OBD)
interfaces of the same vehicle.
}
 
%Based on our evaluation of GPS accuracy over more than 10,000 miles of 
%driving data


To understand and improve the performance of inertial sensors, 
we develop several 
techniques to detect orientation changes, model mounting stability, 
and improve coordinate alignment accuracy. 
We model the orientation of the smartphone as a cluster
of accelerometer data and 
use moving variance to detect possible orientation changes. 
We use \emph{intra-cluster variance} (ICV) to model mounting stability 
of the smartphone and use x-percentile of the variance 
as an indicator of motion estimation accuracy. 
Our intuition is that loose placement introduces more noises 
into sensor data and leads to less accurate acceleration estimation. 
We also find that existing coordinate alignment techniques produce
less accuracy when there are (even small) slopes due to the gravity
components sensed by accelerometer. 
We use an example trip to show that slope may cause coordinate misalignment 
and acceleration over/under estimation. 
Therefore, we use IMU sensors to tracked the slope gradients and subtract
the gravity components sensed by accelerometer. 


We conclude that the performance of inertial sensors are constrained due to
road conditions or human interactions and only well-tuned algorithm can produce higher accuracy 
than GPS in low speed scenarios, 
while slope has limited impact on steering motion estimation performance by gyroscope. 
In addition, we found that, the $90$-percentile accuracy of GPS is under $0.5m/s^2$, 
which indicate a better accuracy than accelerometer on average. 
In particular, we find that GPS is more accurate in high speed scenarios and 
only well-tuned inertial sensors can achieve comparable accuracy. 


The second goal of this paper is to use the above observations to develop a combined system
to conduct driving analytics, called DriveSense, a mobile application that achieve
higher estimation accuracy due to slope awareness, and leverages GPS to complement
inertial sensors, when the latter lacks necessary accuracy and not available. 
More specifically,
DriveSense is able to select the current best estimation 
based on confidence value from GPS and inertial sensors. 
By considering both inputs from GPS and inertial sensors, 
DriveSense has the $75th$ percentile error of $0.2m/s^2$, 
which is $5\times$ better than well-tuned inertial sensors in traditional approach. 


