

Smartphone-based sensing has opened up a whole gamut of applications and services
for road safety, such as driving behavior and driver distraction monitoring. 
One such application is the ability to independently monitor driving behavior --- how well is one
driving the vehicle, e.g., aggressive driving actions, such as rough acceleration, hard brakes, and lane changes, and more.
The popularity of smartphones and built-in Inertial Measurement Unit (IMU) sensors enable
a low-cost way of monitoring such behaviors and actions.
For example, Cambridge Mobile Telematics \cite{cmt} 
develops a smartphone
app to capture driving behaviors and monitor driver
distractions.
The well-known ride-sharing company Uber announced
it will start tracking Uber drivers' driving behaviors
with their smartphones and give them feedbacks that are more detailed than 
the five-star rating customers leave for each driver \cite{uber}.

The general approach to monitoring such vehicle motion (and drive) parameters is to use an accelerometer in a mobile device
 to measure the three-dimensional acceleration
and using a gyroscope to measure the three-dimensional relative rotation speed.
The accelerometer can be used to sense vehicular speed \cite{hansenspeed}
by eliminating estimation errors at reference points where 
ground truth vehicular speed is known.
The gyroscope can be used to detect turns to determine
driver phone use by comparing centripetal accelerations with 
a reference point \cite{wang2013sensing}, 
and it can also be used to track drivers' 
other steering activities \cite{chen2015invisible}.
Calculating the exact orientation of the phone is an important step in enabling estimation of such motion parameters.
Prior work has dealt with different variants of this problem ---  the absolute coordinates estimation problem (orientation relative
to flat earth surface)  \cite{zhou2014use}
and the relative coordinates estimation problem (orientation relative to the orientation of the vehicle which depends on 
the grade and slope of the road) \cite{wang2013sensing, chen2015invisible}.
Both are challenging tasks due to the low accuracy 
and accumulated errors of IMU sensors \cite{zhou2014use}.
Most in-vehicle applications use the relative coordinates estimation 
and depend on coordinate alignment,
which essentially aligns the coordinates of the smartphone to the car \cite{wang2013sensing, chen2015invisible, hansenspeed}. 
In our evaluation, we found that such prior work is not able to accurately estimate
road slope gradients which leads to increased errors in estimating vehicle motion
parameters. Further, the coordinate alignment process usually needs time to
converge and a fast convergence time is desirable in many real-time settings.

\iffalse
But none of those work addressed the problems caused by road slope
and the method to estimate slope gradients before and after coordinate
alignments.
Also, it is still an open question that how fast 
alignment algorithm can converge, e.g., if an algorithm requires the whole
trip to train the algorithm, then it can not be used in real-time applications for short trips.
\fi


In this section, we study the impact of road slopes on estimating various vehicle motion
parameters, especially the acceleration related ones. This is the central problem for
vehicle motion sensing using such mobile devices because many roads are never perfectly
level and even gentle slope of roads can lead to significant inaccuracies.
Investigating this further, we find that there
are three related issues.
First, we use a real-world driving trace to show that
even a perfectly aligned smartphone may overestimate or
underestimate brakes and accelerations due to the gravitational
effect.
When a car is parked on a slope, the linear acceleration
of the heading direction should be zero, 
but the accelerometer will 
sense the gravity component  in the same direction and conclude
the car is accelerating or decelerating on upslope or downslope. 
Therefore, the heading direction component of the accelerometer may be overestimated or underestimated.
Second, we also show that the gravitational effect can prolong the training time
of coordinate alignment and lower alignment accuracy. 
So, it is necessary to estimate the slope gradients even before
coordinate alignment is fully conducted.
Third, estimating slope gradients by gyroscope is challenging due
to the low accuracy of the IMU sensors \cite{zhou2014use}.


To overcome these challenges, we designed a software module that 
can reduce coordinate alignment training time by utilizing selective
training and combining multiple training segments with different weights. 
We also improve coordinate alignment accuracy by 
improving slope gradient estimation at coordinate alignment time.
We believe reducing training time can enable real-time applications like 
aggressive driving behavior warning.
To estimate slope gradients after alignment, 
we use sensor fusion and opportunistic calibration.
Provided that no alignment is perfect, 
we eliminate errors of gyroscope and accelerometer caused by
misalignment.
We only use calibration segment,
which refers to the road segments that accelerometer can produce accurate gradient estimation,
to calibrate gyroscope. 
Compared to the state of the art, our coordinate alignment module can converge much 
faster and our linear acceleration estimation module can achieve better accuracy. 


\nop{
\textbf{Contributions}. We design a smartphone-based system to improve the accuracy of estimating
vehicle motion parameters. 
We illustrate that road gradients, even gentle ones,
can cause acceleration overestimation or underestimation 
due to the gravitational effect.
They also have a great impact on coordinate alignment. 
We implement a software module that can reduce coordinate alignment
training time to enable real-time estimation of these parameters.
It can also estimate slope gradient before and after coordinate alignment
by combining the estimation results from accelerometer and gyroscope. 
By removing the gravity components, it is able to capture vehicle motion parameters
with greater accuracy than the state of the art,  
comparing with ground truth speed readings from the 
in-vehicle on-board diagnostics (OBD) port \cite{obd}.
The coordinate alignment is faster than the existing state of the art alternatives.
We evaluate our solution by using traces across highways and urban city roads of about 10,000 miles.
}





