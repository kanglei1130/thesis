
\textbf{Extra Power Consumption of GPS}. 
The use of GPS will accelerate the power consumption of the smartphone. 
We believe user experience is more important
than power consumption within vehicle, 
where the user is able to charge the smartphone. 
We find that some vehicle travel recording applications, like DriveWell \cite{cmt} , monitoring GPS information
in the background. 
We believe the users are willing to enjoy some 
vehicle service such as navigation and travel recording
in exchange for some extra power.  



\textbf{Better or Worse GPS accuracy}. 
Our GPS evaluation is limited to existing techniques as well as the
geographic scope where we collected the data. 
The GPS has a chance to perform better with more advanced algorithms
or worse with signals blocked by high buildings. 
Google has implemented the API which allow the application layer to use raw GNSS data as input
since Android 7.0 system \cite{android_gnss, google_gnss_tools}.
With extra input, we believe the accuracy of GPS in commodity 
smartphone could be improved. 
The accuracy of GPS in metropolitan cities could be decreased when vehicles are passing through high buildings blocking the GPS signal.
DriveSense is a system framework that can alternatively use the best estimation
Method.
With the training and modeling techniques presented in this paper, 
we believe DriveSense is able to accommodate new techniques.  

\textbf{High-end Vehicles}. 
There are emerging selfdriving techniques that ultilize 
expensive hardwares to scan the road, 
which may provide better accuracy on road condition
monitoring. 
Also they could be equipped with built-in sensors
and alignment is not a problem anymore. 
But our techniques about linear acceleration
estimation and fusion with GPS still work
in such cases. 
We argue that the majority of existing
vehicles are still low-end vehicles 
and we expect they will still dominate 
the road in the next decade. 
Our techniques can be used in either high-end 
or low-end vehicles by using only
a single smartphone. 


\textbf{Device and Orientation Diversity}. 
We are not able to try all the devices and 
every possible orientation. 
The devices we used include Motorola Xoom,
HTC Hero, LG Nexus 5, LG Nexus 5x and Motorola Nexus 6. 
The devices are placed in the back pocket of passenger
seat, vertically /horizontally in car mount holder, 
car cup holder, passenger seat, pants pocket, 
and passenger hand.  
We believe the diversity of devices and orientations in our study are representative enough. 
We have not observed neither particular device nor 
the orientation influence on vehicle motion sensing
with inertial sensors. 



