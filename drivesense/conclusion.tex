


Smartphones are commonly used for driving analytics applications. 
Traditional approaches assume experimental
cases where the smartphone is stably mounted with fixed relative orientation
and the vehicle is travelling on flat roads. 
By using an example experiment, we show that 
even perfectly aligned accelerometer suffers acceleration overestimation or underestimation, 
which is caused by gravitational force, misalignment and slope estimation error.  
Moreover, the accuracy of IMU sensors are sensitive to human interactions
as well. 
For example, frequent relative orientation change and less stable mounting
may cause significant estimation errors. 
In this paper, the defects are remedied
with following innovative techniques. 
First, a slope-aware alignment algorithms to reduce the slope influence, 
meanwhile, to improve alignment accuracy. Also, we track the linear 
acceleration of the vehicle to address acceleration over/under estimation problems. 
Second, in order to timely update 
vehicular motion parameters, the relative orientation changes of 
smartphone are detected using machine learning algorithms. 
Third, we model mounting stability of the smartphone and 
evaluate the sensing accuracy under different stability estimations.  
Fourth, we present the tradeoffs between inertial sensors, the primary sensors, 
and GPS, which is used when inertial sensors lose accuracy or disable. 
To evaluate our solutions, we compare the estimated 
accelerations with those calculated from OBD speed readings.  
The accuracy of our methods are evaluated with highway and urban traces
covered by 13,000 miles in the northwestern US.
We show through our experiments and analysis that
compared to current state of the art techniques, 
our method improves the $75$-percentile accuracy 
by $5\times$ comparing with well-tuned inertial sensors in traditional approach. 



