
In this section, we discuss some design considerations of EcoDrive. 


\textbf{Hybrid Vehicle (HV) and Electric Vehicle (EV)}. 
EcoDrive is designed for gasoline-powered vehicles, but the modeling process
and gas pedal emulation can also be applied to HVs and EVs. 
Currently, the majority of vehicles running
on street are still gasoline-powered vehicles. 
It is important to develop systems like EcoDrive
to work on regular vehicles to improve fuel efficiency
and limit carbon pollution. 
The core of EcoDrive is that it can work on regular vehicles
with easy and recoverable installation. 


\textbf{Instant Fuel Economy Display}.
In urban environments, acceleration is not fuel efficient, but accelerating to a higher speed
in a careful way can improve fuel efficiency. 
If the driver refuses to accelerate due to low instant fuel efficiency,
the car will drive in low fuel efficient speed and eventually increase
fuel consumption. 
Similarly, releasing gas pedal can increase instant fuel efficiency dramatically, 
but frequent acceleration and deceleration will consume more fuel
than cruising under certain speed. 
Therefore, instant fuel economy display is misleading
and may reduce overall fuel efficiency. 



\textbf{Impact of Traffic and User Experience}.
EcoDrive controls the vehicle in a way that is similar to cruise control, 
so the driver should keep a safe distance to
front vehicle. 
Since EcoDrive accelerates the vehicle in various ways according to
different road segment lengths, 
it is more challenging for human drivers to keep safe distance than cruise
control in dense traffic scenarios, because
drivers need to frequently brake to avoid front-end collision. 
Therefore, EcoDrive is more suitable for drivers to use in low traffic volume scenarios. 
However, it is shown from the data we collected that drivers tend to
accelerate much more aggressive than the optimal acceleration patterns. 
In other words, EcoDrive achieves higher fuel efficiency by accelerating
slower than most drivers usually do. 
Also, EcoDrive uses fixed acceleration pattern for fixed road length
and speed limit, 
drivers and passengers can easily get used to the driving style
of EcoDrive. 
Therefore, we expect EcoDrive can be useful in most urban driving scenarios 
(except traffic hours) and drivers can still feel that the vehicles are under control
due to predictable accelerations. 
We also expect that EcoDrive can tolerate more complex traffic
conditions when integrating with front object detection and
route planning systems equipped on driverless cars. 


\textbf{Limitations}. 
First, EcoDrive requires the road length as input in short length road segment 
(e.g., shorter than 200m). 
Second, it relies on drivers to brake or switch to disable EcoDrive mode. 
The operation complexity is acceptable for drivers as proven
by cruise control.
Third, EcoDrive requires mileage trainings to build an accurate model, i.e., 
1000 miles driving data can build a very accurate model as shown in evaluation. 
Some optimization can be made to reduce training time, e.g., sharing the AFR profile 
among same vehicle models.
Fourth, we did not evaluate EcoDrive by end-to-end scenarios in urban environment, 
e.g., what are the fuel savings from home to work. 
There are different traffic volumes 
and different traffic light schedules among different trips. 
Therefore, it is challenging to compare EcoDrive with human drivers 
in such scenarios. 
Given the fuel improvement in arbitrary length road segments, 
we believe EcoDrive can improve fuel efficiency in end-to-end scenarios as well.   

Each trip can be divided into three parts: acceleration, cruising and braking. 
Acceleration and cruising consume most of the fuel during a trip. 
For this work, EcoDrive focuses on the acceleration part and cruising part. 
There is no direct way to fairly compare two brakes that made by different drivers, 
or different brakes made by the same driver. 
Therefore, we did not include the fuel consumption of braking in our evaluation. 

There are other factors that affect fuel consumption, e.g., 
tire pressure, temperature and wind direction etc. 
A more accurate model can be built by using more 
vehicle parameters. 
However, there is no direct way to access some in-vehicle parameters
like tire pressure. 
We advocate that the vehicle manufacturers provide more
information through OBD port that can be used to enable
more applications. 
Some other real world parameters, 
e.g., traffic light schedule, stop sign and wind direction etc., 
can also be used to improve fuel efficiency. 
We believe such extra information can be used to improve 
EcoDrive in broader aspects.


