

\subsection{Cruise Control}

%\textbf{Cruise Control}. 
Cruise control \cite{cruise_control} is a system that automatically
controls the vehicle to drive in a certain speed. 
\cite{bengtsson2001adaptive} introduces Adaptive Cruise Control (ACC)
that adjusts vehicle speed based on distance to cars ahead
by using radar. 
\cite{lin2009effects} studies the effects of time-gap
settings of adaptive cruise control on driving performance
in a bus driving simulator. 
\cite{loos2011adaptive} studies a distributed car control system
that every car is controlled by adaptive cruise control. 
\cite{ioannou1993autonomous} develops an intelligent cruise control system 
for automatic vehicles and verify the effectiveness of the system
under emergency situations by simulation.  
EcoDrive is more fuel efficient than adaptive cruise control 
in two ways. 
First, EcoDrive is able to improve fuel efficiency by 
adjusting speed according to road conditions and vehicle types. 
Second, EcoDrive tracks the fuel efficiency under different cruising
speeds and provides a tradeoff between fuel efficiency and travel time. 

\subsection{Fuel Efficiency Enhancement}

%\textbf{Fuel Efficiency Enhancement}. 
Increasing vehicular fuel efficiency has been a topic 
of much recent research \cite{ganti2010greengps, koukoumidis2011signalguru, 
boriboonsomsin2010eco, tulusan2012providing}.
GreenGPS \cite{ganti2010greengps} requires a fuel map server
to record the statistical route fuel usage and calculate the most
fuel efficient route for drivers. 
SignalGuru \cite{koukoumidis2011signalguru} requires vehicles sharing
traffic light information so that drivers can adjust the speed of
the vehicle to avoid stops in front of traffic lights. 
\cite{boriboonsomsin2010eco} uses an on-board eco-driving device 
that provides instantaneous fuel economy
feedback affects driving behaviors, and consequently fuel economy. 
\cite{tulusan2012providing} evaluates the impacts of a smartphone application
on driving behaviors to enhance fuel efficiency. 
Different from existing approaches that are focusing on advising human drivers to adjust driving behaviors, 
EcoDrive is designed to assist human drivers.
Both \cite{lang2013opportunities} and \cite{lang2014prediction} show that some reduction of fuel consumption can be achieved
by vehicle-to-vehicle communication. 



\subsection{Vehicle Dynamics Sensing}

%\textbf{Vehicle Dynamics Sensing}. 
Smartphones equipped with various sensors are widely used
to sense vehicle dynamics to understand driving behaviors and road conditions.
\cite{johnson2011driving} detects aggressive driving behaviors 
based on accelerometer sensor readings from smartphone. 
\cite{wang2013sensing} determines driver phone use by comparing
the centrifugal force of smartphone with that of a fixed reference sensor
during turns. 
\cite{eriksson2008pothole} monitors road surface conditions by using accelerometers. 
\cite{Mohan2008Nericell} detects various road and traffic conditions, 
e.g., potholes, bumps, braking, and honking, 
by using various sensors in smartphones. 
\cite{progressive} senses speeds from the OBD port to identify aggressive
driving behaviors and offer discounts for drivers with good driving habits.  
However, none of these work models vehicle dynamics as functions of instant fuel consumption. 

\subsection{Vehicle Force Modeling}

%\textbf{Vehicle Dynamics Modeling}. 
\cite{andersson2012online} models rolling resistance and air drag based on some measurements 
on the vehicle, e.g., effective areas to model air drag. 
\cite{zanasi2001dynamic} models vehicle transmission system based on gear sizes. 
\cite{canudas1999dynamic} models tire friction based on tire parameters, e.g., radius. 
\cite{brusamarello2010dynamic} presents an electronic system attached on wheel to measure and
monitor of the torque transmitted to the wheel of a moving vehicle.
\cite{vong2006prediction} models engine torque and horsepower by RPM when the
engine is not connected with transmission. 
Our model uses the parameters available on car CAN bus to find the relations between fuel consumption and vehicle speed changes.
Different from existing work, 
our model does not rely on the detail physics properties of the car, 
e.g., tire radius and transmission gear size etc.


\subsection{Autonomous Driving}

%\textbf{Autonomous Driving}. 
There have been major advances in designing and
building autonomous vehicle operating systems to 
realize safer and more convenient vehicles 
\cite{googledriverlesscar, urmson2008autonomous, litman2013autonomous, kim2013towards}. 
These work focuses on issues related to object detection and efficient navigation and so forth. 
CarSpeak \cite{kumar2012carspeak} is designed as a communication system to share sensory
data between autonomous vehicles for obstacle detection.
Different from autonomous cars studies, 
we focus on the design and implementation of 
fuel-aware acceleration control system that works on existing regular vehicles. 
We expect that EcoDrive can be integrated with autonomous driving cars, 
but EcoDrive can also be used independently. 
Also, EcoDrive can work on existing vehilces with little hardware changes. 



 

